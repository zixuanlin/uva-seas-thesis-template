%% ***************************************************************************************************************************************************************
%%  Adapted from: Dissertation template and document class for Princeton University 
%%                    (Author: Jeffrey Scott Dwoskin <jdwoskin@princeton.edu>)     
%% ***************************************************************************************************************************************************************


%% ======================Choose "Printed" or "ProQuest" or "Online" or "Draft" (uncomment commands under the option you choose)====================
%% Printed: single space, no hyperlinks (Delete all output files before changing this mode -- it will turn hyperref package on and off)
% \documentclass[12pt, lot, lof, glossary, singlespace]{puthesis}
% \newcommand{\printmode}{}

%% ProQuest: double space, outlined links
\documentclass[12pt, lot, lof, glossary]{puthesis}
\newcommand{\proquestmode}{}

%% Online: double space,  hyperlinks
% \documentclass[12pt, lot, lof, glossary]{puthesis}

%% Draft: no "List of Tables", "List of Figures", .etc (also see the "ifodd" command below to disable more frontmatter)
% \documentclass[12pt]{puthesis}

%% ========================================================================info===================================================================
\title{Thesis/Dissertation Title}
\author{Author Name}
% \submitted{}  % degree conferral date
% \copyrightyear{}  % year in which the copyright is secured by publication of the dissertation.
% \adviser{}  
% \department{}

%% ========================================================= Printed, ProQuest, Online formatting=================================================
\ifdefined\printmode     % [Printed] (single space, no hyperlink)
  \usepackage{url}     % url package understands urls (with proper line-breaks) without hyperlinking them
\else
  \ifdefined\proquestmode % [ProQuest] (double space, outlined links instead of colored links)
    \usepackage{hyperref}
    \hypersetup{bookmarksnumbered}
    
    %% copy the already-set title and author to use in the pdf properties
    \makeatletter
    \hypersetup{pdftitle=\@title,pdfauthor=\@author}
    \makeatother
  
  \else % [Online] (makes links by coloring the text)
    \usepackage{hyperref}
    \hypersetup{colorlinks,bookmarksnumbered}
    
    \makeatletter
    \hypersetup{pdftitle=\@title,pdfauthor=\@author}
    \makeatother
    
    %% make the page number rather than the text be the link for ToC entries
    % \hypersetup{linktocpage}
  
  \fi
\fi

%% ==============================================================================rename============================================================
\renewcommand*{\contentsname}{Table of Contents}
\renewcommand{\bibname}{References}

%% =============================================================================glossary===========================================================
\usepackage[acronym]{glossaries}
\makeglossaries
\newacronym{glo-label1}{ABC}{A... B... C...}

\newglossaryentry{glo-label2}
{  name=term,
   description={discriptions}
}


  
%% ===========================================================================Front-matter=========================================================
\ifodd 0 % no titlepage, no copyrightpage, no abstract, no acknowledgement, no dedication
  \renewcommand{\makecoverpage}{}
  \renewcommand*{\makecopyrightpage}{}
  \renewcommand*{\makeabstract}{}
  
\else % all have
  \abstract{ Write your abstract here. 

Your abstract can be any length, but should be a maximum of 350 words for a Dissertation for ProQuest's print indicies (or 150 words for a Master's Thesis); otherwise it will be truncated for those uses~\cite{proquest2006}.

 }
  \acknowledgements{ I would like to thank the Department of ... for providing ... I would like to thank my parents, without whom my life would not be possible. I would also like to thank my advisor, my dissertation committee, and my research collaborators because every graduate student needs to do so. And finally, I thank the members of my research group, to whom ...  

Don't forget to ask your advisor if your work was sponsored by a grant that needs to be acknowledged in this section.  

 }
  \dedication{To my parents.}
\fi 

%% =================================-======================================Hide some chapters======================================================
%% If you want to produce a pdf that includes only certain chapters, specify them with includeonly, in addition to including all chapters below.
% \includeonly{ch-intro/chapter-intro}

%% You can also specify multiple chapters.
% \includeonly{ch-intro/chapter-intro,ch-usage/chapter-usage}
% \includeonly{chap1,chap2,chap3}

%% ===================================================================================
%%                                  Notes
%% ===================================================================================
%% Footnotes should be placed after punctuation.\footnote{place here.}
%% Generally, place citations before the period~\cite{anotherauthor}.
%% The proper usage for i.e., and e.g., include commas ``(e.g., option A, option B)''


%%%%%%%%%%%%%%%%%%%%%%%%%%%%%%%%%%%%%%%%%%%%%%%%%%%%%%%%%%%%%%%%%%%%%%%%%%%%%%%%%%%%%%%%%%%%%%%%%%%%%%%%%%%%%%%%%%%%%%%%%%%%%%%%%%%%%%%%%%%%%%%%%%%%%%%%%%%%%%%%%%
%%-------------------------------------------------------------------------------------BEGIN----------------------------------------------------------------------
\begin{document}

\makefrontmatter
%% If you've disabled frontmatter, you can insert the toc manually
% \tableofcontents\clearpage

%% \include lets us split up the document (and each include starts a new page)
\chapter{Introduction\label{ch:intro}}

\section{Background}

\acrfull{glo-label1}

\GLS{glo-label2}

%%
% \gls, \Gls, \glspl, \glspl
% \acrlong, \acrshort, \acrfull
\input{ch-intro/overview}


\chapter{Chapter Name\label{ch:2}}

\section{Topic 1}
Contents of topic 1...


\section{Topic 2}
Contents of topic 2...




\chapter{Discussion and Future Work\label{ch:discussion}}

\input{ch-discussion/discussion}
\section{Future Work}

Future work should include ... It should also ...





%% ---------------------------------------------------------------------------------References--------------------------------------------------------------------
\singlespacing     % Make the bibliography single spaced
\bibliographystyle{unsrt} 

%% add the Bibliography to the Table of Contents
\cleardoublepage
\ifdefined\phantomsection
   \phantomsection  % makes hyperref recognize this section properly for pdf link\else
 \fi
\addcontentsline{toc}{chapter}{References}

\bibliography{references} % include your .bib file

%% ---------------------------------------------------------------------------------Appendix-----------------------------------------------------------------------
\appendix 
%% all chapters following will be labeled as appendices
\include{ch-appendices/implementation}
\chapter{Usage\label{ch:usage}}

This \LaTeX{} template and document class \texttt{puthesis.cls}, were adapted from ``Dissertation template and document class for Princeton University''. The Mudd Library website~\cite{mudd2009} provides detailed specifications for how to format your disseration~\cite{muddthesis2009} (for Princeton use). Also, review the ProQuest Dissertation Guide~\cite{proquest2006}, which has additional formatting rules that are important for the submission of the electronic copy of your dissertation. The Princeton University template was created in 2010 by Jeffrey Dwoskin, and adapted from a template provided by the math department. Their original version is available at: \url{http://www.math.princeton.edu/graduate/tex/puthesis.html}

This is \textbf{NOT} an official document. Please verify your school and department requirements before using this template or document class.

\section{Figures}
\label{sec:usage:figures}

Everyone needs floating figures in their dissertation. 

As shown in Figure~\ref{fig:usage:titlepage}, the Mudd Library dissertation requirements~\cite{muddthesis2009} specify additional options for formatting the title page. For example, if your thesis has multiple volumes, or to indicate the proper formatting for a master's thesis.

\begin{figure}[htb]
  \begin{center}
    \includegraphics[width=0.9\linewidth]{ch-appendices/usage/figures/titlepage}
    \caption[Sample Title Page Layout]{Sample title page layout~\cite{muddthesis2009}}
    \label{fig:usage:titlepage}
  \end{center}
\end{figure}


\input{ch-appendices/usage/table}
\chapter{Usage\label{ch:usage}}

This \LaTeX{} template and document class \texttt{puthesis.cls}, were adapted from ``Dissertation template and document class for Princeton University''. The Mudd Library website~\cite{mudd2009} provides detailed specifications for how to format your disseration~\cite{muddthesis2009} (for Princeton use). Also, review the ProQuest Dissertation Guide~\cite{proquest2006}, which has additional formatting rules that are important for the submission of the electronic copy of your dissertation. The Princeton University template was created in 2010 by Jeffrey Dwoskin, and adapted from a template provided by the math department. Their original version is available at: \url{http://www.math.princeton.edu/graduate/tex/puthesis.html}

This is \textbf{NOT} an official document. Please verify your school and department requirements before using this template or document class.

\section{Figures}
\label{sec:usage:figures}

Everyone needs floating figures in their dissertation. 

As shown in Figure~\ref{fig:usage:titlepage}, the Mudd Library dissertation requirements~\cite{muddthesis2009} specify additional options for formatting the title page. For example, if your thesis has multiple volumes, or to indicate the proper formatting for a master's thesis.

\begin{figure}[htb]
  \begin{center}
    \includegraphics[width=0.9\linewidth]{ch-appendices/usage/figures/titlepage}
    \caption[Sample Title Page Layout]{Sample title page layout~\cite{muddthesis2009}}
    \label{fig:usage:titlepage}
  \end{center}
\end{figure}


\input{ch-appendices/usage/table}
\chapter{Usage\label{ch:usage}}

This \LaTeX{} template and document class \texttt{puthesis.cls}, were adapted from ``Dissertation template and document class for Princeton University''. The Mudd Library website~\cite{mudd2009} provides detailed specifications for how to format your disseration~\cite{muddthesis2009} (for Princeton use). Also, review the ProQuest Dissertation Guide~\cite{proquest2006}, which has additional formatting rules that are important for the submission of the electronic copy of your dissertation. The Princeton University template was created in 2010 by Jeffrey Dwoskin, and adapted from a template provided by the math department. Their original version is available at: \url{http://www.math.princeton.edu/graduate/tex/puthesis.html}

This is \textbf{NOT} an official document. Please verify your school and department requirements before using this template or document class.

\input{ch-appendices/usage/figure}
\input{ch-appendices/usage/table}
\input{ch-appendices/usage/usage}







\end{document}

